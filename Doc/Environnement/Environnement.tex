%=======================================================================
\documentclass[12pt,a4paper]{article} 

\usepackage[dvips]{graphicx}
\usepackage{ifpdf}
\usepackage[dvips,
bookmarks,
bookmarksopen,
backref,
colorlinks,linkcolor={blue},citecolor={blue},urlcolor={blue},
]{hyperref}

\setlength{\paperwidth} {21cm}
\setlength{\oddsidemargin} {0.cm}
\setlength{\textwidth} {16cm}
\setlength{\marginparsep} {0cm}
\setlength{\marginparwidth} {0cm}
\setlength{\headheight} {0.0cm}  
\setlength{\textheight} {24.cm}


%=======================================================================

\title{EOS installation guide EOS 1.5.0}

%=======================================================================

\begin{document}

\maketitle
%\newpage
\vspace{2.cm}
\tableofcontents

\newpage
%
%*************************************************
\section{Introduction}
%*************************************************
%
The root directory of your distribution is the directory "eos", called
\$NEPTUNE\_ROOT thereafter.
This file has been produced from: \\
\$NEPTUNE\_ROOT/Doc/Environnement/Environnement.tex

%
%*************************************************
\section {Process of EOS installation}
%*************************************************
%
At first the configure stage attempts to define values for the variables used
during compilation and installation. The configure stage uses shell-script
and CMAKE.  This configure stage adds the plugins if needed and build 
automatically the Makefile in each required directory. 

The compilation is the second stage, at the end of this stage the EOS library
is made. The doxygen documentation is also made.

Then installation and test are the next stages. 

The summary of the process in shell langage is :
\begin{verbatim}
  cd eos
  ./configure {options}
  make
  make install
  make check
\end{verbatim}

%*************************************************
\section {Configure Options}
%*************************************************

For an overview of the options, use:
\begin{verbatim}
  ./configure --help
\end{verbatim}


\subsection {Supply options by argument or by file}
%---------------------------
The options may be supplied to the configure script by argument or by
file. By default, this file is \verb|user_env.txt| and it is located in
\$NEPTUNE\_ROOT directory.  You can use another location and another 
file's name by using:
\begin{verbatim}
  ./configure --user-env-file=<path>/<file>
\end{verbatim}
For example, you can write the line \verb|--prefix=<path>| in this file
instead of using \verb|configure --prefix=<path>|.
The arguments of this file are read before the configure command
arguments. Therefore, if you use \verb|configure --prefix=<path2>|, the
\verb|<path2>| is used instead of the path defined in this file.


\subsection {Location of installed files}
%---------------------------
By default, the installation made with \verb|make install| will install
the package's files in \$NEPTUNE\_ROOT.  You can specify an other 
installation directory by using the configure option :
\begin{verbatim}
--prefix=<path>.
\end{verbatim}



\subsection {Compilers}
%---------------------------
The compilers are guessed at the configure stage by cmake.
You can change the default C++, C or fortran compiler by
using the configure options :
\begin{verbatim}
 ./configure --cxx=<my_c++_compiler>
             --cc=<my_c_compiler> 
             --fc=<my_fortran_compiler>
\end{verbatim}

You can also override the choices made by configure using environment variables :
\begin{verbatim}
 CXX=<my_c++_compiler> CC=<my_c_compiler> FC=<my_fortran_compiler> ./configure
\end{verbatim}
Nevertheless, with cmake, you must remove the file CMakeCache.txt if
the configure has already been done. This method to force the
compilers is not recommanded.

General compilation options may be added with  \verb|--add_cppflags=<opt1>,<opt2>,...|



\subsection {Libraries}
%---------------------------
Two libraries are built : a dynamic library libshared\_eos and
a static library libstatic\_eos.a


%*************************************************
\section {Thermodynamic Models as Plugins}
%*************************************************
Four Thermodynamic Models (T.M.) are available as plugins:
- Refprop-9.1
- Cathare-1.5.b  from  1.5.b release of CATHARE2 code
- Cathare2       from  last version  of CATHARE2 code
- Thetis-2.1.b

To use them, there are four options to add in stage configure :
\begin{verbatim}
--with-cathare=<path>/<file.tgz>/<file.tar.bz2>
--with-cathare2=<path>/<file.tgz>/<file.tar.bz2>
--with-refprop=<path>/<file.tgz>/<file.tar.bz2>
--with-thetis=<path>/<file.tgz>/<file.tar.bz2>
\end{verbatim}



%*************************************************
\section {Interpolator}
%*************************************************

Interpolator is build with the configure option --with-interpolator
Prerequisite softwares HDF5 and MED-3 are compulsory


%*************************************************
\section {EOS GUI}
%*************************************************

EOS GUI is build by default or when \verb|--with-gui| is given in configur stage.
For EOS GUI, some softwares are required : SWIG, QT4, pyQt, python-devel,
gnuplot. Two softwares are optional : HDF5, MED-3

The configure option  --without-gui  diseables the EOS GUI.

There are two methods to build EOS GUI :
\begin{itemize}
\item with prerequisite softwares HDF5 and MED, the configure options 
  \verb|--with-gui --with-hdf5={path_hdf5} and -with-med={path_med}| are required
\item without HDF5 and MED, nothing to do except to define \verb|--with-gui| configure 
  option
\end{itemize}

In order to run the EOS GUI, you can use the eos\_gui script located in the bin
directory or use the eos script with the option gui : "eos gui"




%*************************************************
\section {Prerequisites}
%*************************************************



\subsection {MPI}
%---------------------------
For some operating systems (O.S.), HDF5 software is installed in parallel mode 
with MPI; on this O.S. and for MPI library non standard installation a new
optional option \verb|--with-mpi={path_mpi}\verb|  is introduced, and also derivatives
options \\ \verb|--with-mpi-lib={path2_mpi}/lib| 
           \verb|--with-mpi-include={path3_mpi}/include|


\subsection {SWIG}
%---------------------------
Non standard SWIG installation requires a new option
\verb|--with-swig-exec={fic_swig}|



%
%*************************************************
\section{API python}
%*************************************************
%

The EOS python API is not yet available.
The configure option \verb|--with-api-python| is required. Some
products are needed : SWIG, python-devel.




%
%*************************************************
\section{Documentation}
%*************************************************
%
The main documentation page is in \$NEPTUNE\_ROOT/index.html or in \\
\$NEPTUNE\_PREFIX/Doc/index.html.
One part of the documentation is generated by doxygen (used by default)
and the other part is build with latex (\verb|--with-doc-latex|, unused by
default). 

\end{document}
