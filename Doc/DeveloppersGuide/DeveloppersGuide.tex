\documentclass[12pt]{article} 

\title{Developpers Guide.}
%\author{Philippe Emonot}

\begin{document}
\maketitle
\newpage
\tableofcontents
\newpage
%-----------------------------------------------
\section{Test d'installation généralisée de eos}
%-----------------------------------------------

Il est possible de tester l'installation de eos sur de nombreuses
machines à l'aide des scripts localisées dans le répertoire
eos/portability.

L'instruction de lancement est {\it sh test0.sh} et les résultats sont
synthétisés dans une fichier html : nuit\_eos.html. La liste des
machines et les options asociées pour chaque machine cible sont
définies dans le script shell test0.sh.

Le script shell test0.sh permet :
\begin{itemize}
 \item de créer l'archive eos.tar qui sera testée sur touts les machines cibles,
 \item de lancer l'installation sur une liste de machines cibles avec
   les options qui vont bien pour chaque machine.  
 \item de lancer la concaténation de la fiche de synthèse nuit\_eos.html
\end{itemize}

Le script shell distrib.sh permet, pour chaque machine cible :
\begin{itemize}
 \item de vérifier si la machine est acessible,
 \item de copier l'archive sur la machine cible,
 \item de détarer l'archive sur la machine cible,
 \item de lancer les scripts prepare.sh, configure.sh, make.sh,
   make\_check.sh et make\_install sur la machine cible
 \item d'écrire le compte-rendu de l'installation dans le fichier de synthèse nuit\_eos.html
\end{itemize}

Le script shell prepare.sh permet, sur la machine cible sur laquelle il
est lancée :
\begin{itemize}
 \item de détarer l'archive eos.tar.gz
\end{itemize}




%-----------------------------------------------
\section{Installation}
%-----------------------------------------------

A partir de La version 1.5 il n'y plus qu'une seule méthode d'installation, la méthode de référence utilisant cmake.
% \item une méthode alternative à l'utilisation de cmake se basant sur
%   la méthode d'origine d'installation,
% \item la méthode d'origine d'installation, à ne plus employer de préférence.

Cette installation permet de relancer certaine étape du configure grâce aux commandes suivantes :
\begin{itemize}
\item make doc-doxygen : générer la documentation doxygen si l'option --with-doc-doxygen a été levée au configure ;
\item make doc-latex : générer la documentation latex si l'option --with-doc-latex a été levée au configure ; 
\end{itemize}
%\subsection{Installation avec cmake}
%-----------------------------------------------
%\subsection{Installation sans cmake}
%-----------------------------------------------
%{\bf fonction read\_arg}
%\begin{itemize}
% \item lit les options,
% \item détermine quelle partie de NEPTUNE\_install doit être exécutée
% \item définit \$eos\_prefix,\$eos\_arch,
% \item écrit un fichier .eos\_env\_arg 
%\end{itemize}
%
%{\bf configure}
%\begin{itemize}
% \item écrit le fichier de configuration
%   \$eos\_env\_file=.Neptune.bashrc dans le repertoire build
%   incluant des options passées en argument et des options par défaut.
% \item source l'enviromment
% \item efface les précédents fichiers .out et .err s'ils existaient,
% \item def\_prefix : définit les variables d'environnement associées
%   au répertoire pour l'installation, 
% \item info\_env : écrit des infos sur les paramètres, dont des infos
%   pour renseigner le tableau nuit\_eos.html.
% \item installe les plugins,
% \item définit le contenu de config.hxx
% \item création de l'API de eos
% \item construit les makefile
%\end{itemize}

\end{document}
